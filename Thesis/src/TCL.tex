\chapter{Temporal Cooperation Logic}
\section{TCL}
TCL is an extension of BSIL which allow SQs appear after temporal operators.
Thus, TCL is able to describe following complicate properties while BSIL is not.

Let's reuse the example of prisoner dilemma\ref{exmp.pd}.
We can specify that prisoner 1 has a {\em nice} strategy which means he will not betray other players until been betrayed in TCL by following expression.
\begin{center} 
\hfill 
$\langle 1\rangle \pfrr ((\langle+\rangle \nxt \neg\mbox{\tt betray}_1)
\vee \bigvee_{a\neq 1}\mbox{\tt betray}_a)$
\hfill (I) 
\end{center}

Similarly, we can describe more involved strategies, such as 'Prisoner 2 will always betray when she does not have the power to force Player 1 to always play nice.'
\begin{center} 
\hfill 
$\langle 2\rangle (
(\langle+\rangle \pfrr \mbox{\tt betray}_2) 
\vee \langle+\rangle\pfrr ((\langle+\rangle \nxt \neg\mbox{\tt betray}_1)
\vee \bigvee_{a\neq 1}\mbox{\tt betray}_a))$
\hfill (J) 
\end{center} 

Moreover, we can describe deterministic Nash equilibria like 'the three agents can cooperate such that every agent either eventually leaves prison, or stays forever in prison regardless of her own strategy under the current strategies of the remaining prisoners' in TCL.
\begin{center}
\hfill
$\langle 1,2,3\rangle
        \bigwedge_{a\in [1,3]}\big((
        \langle+\emptyset\rangle\pevt \neg \mbox{\tt jail}_a) \vee
        \langle-a\rangle \pfrr \mbox{\tt jail}_a\big)$
\hfill (K)
\end{center}


\subsection{Syntax}
A TCL formula $\phi$ is constructed with the following three syntax rules.
\smallskip

\noindent
% \begin{center}
$\begin{array}{rcl}
\phi    & ::= & p\;|\; \neg\phi_1 \;|\; \phi_1\vee \phi_2
    \;|\; \langle  A\rangle \psi
    \\
\psi  & ::= & \phi \;|\; \eta \;|\; \psi_1\vee \psi_2\;|\; \psi_1\wedge \psi_2
    \;|\; \langle+A\rangle \psi_1
    \;|\; \langle+A\rangle\nxt\psi_1 
    \;|\; \langle+A\rangle\eta_1\until \psi_1
    \;|\; \langle+A\rangle\psi_1\rmrel \eta_1 \\
    & | & \langle-A\rangle \psi_1
    \;|\; \langle-A\rangle\nxt\psi_1 
    \;|\; \langle-A\rangle\eta_1\until \psi_1
    \;|\; \langle-A\rangle\psi_1\rmrel \eta_1 
    \\
\eta  & ::= & \phi \;|\; \eta_1\vee \eta_2\;|\; \eta_1\wedge \eta_2
    \;|\; \langle+\rangle\nxt\eta_1 
    \;|\; \langle+\rangle\eta_1\until \eta_2
    \;|\; \langle+\rangle\eta_1\rmrel \eta_2 \\
    & | & \langle-A\rangle\nxt\eta_1 
    \;|\; \langle-A\rangle\eta_1\until \eta_2
    \;|\; \langle-A\rangle\eta_1\rmrel \eta_2 
\end{array}$
% \end{center}
\smallskip

Here, $p$ is an atomic proposition in $\calp$ and $A\subseteq \{1,\ldots,m\}$ is an agency.  
Property $\langle A\rangle\psi_1$ is an (existential) 
strategy quantification (SQ) specifying that
there exist strategies of the agents in $A$
that make all plays consistent with these strategies satisfy $\psi_1$.
Property $\langle+A\rangle\psi_1$ is an (existential) 
strategy interaction quantification (SIQ) and 
can only occur bound by an SQ.
Intuitively, $\langle+A\rangle\psi_1$ means that there exist strategies of the agents in $A$
that work with the strategies introduced by the ancestor formulas.
Likewise, $\langle -A \rangle$ indicates a revocation of the strategy binding for the agents in $A$.
$\langle + \rangle$ is an abbreviation for $\langle + \emptyset\rangle$ or, equivalently $\langle - \emptyset \rangle$.
Thus, it neither binds nor revokes the binding of the strategy of any agent.
Yet, it provides a temporalisation in that it provides a tree formula that can be interpreted at a particular point.

% For convenience, we call an SQ also as an SIQ that interacts with 
% no ancestor formula strategies. 

`$\until$' is the {\em until} operator.  
The property $\psi_1\until\psi_2$ specifies a play along which 
$\psi_1$ is true until $\psi_2$ becomes true.  
Moreover, along the play, $\psi_2$ must eventually be fulfilled.   
`$\rmrel$' is the {\em release} operator.  
Property $\psi_1\rmrel\psi_2$ specifies a play along which either 
$\psi_2$ is always true or $\psi_2\until(\psi_1\wedge\psi_2)$ 
is satisfied. (Release is dual to until: $\neg(\phi_1 \until \phi_2) \, \Leftrightarrow \, \neg \phi_2 \rmrel \neg \phi_1$.)

In the following we may use $\langle?A\rangle\psi$ to conveniently 
denote an SQ or SIQ formula with `$?$' is empty, `+', or `-'.  
An SIQ $\langle\pm A\rangle\psi$ is called non-trivial if 
$A$ is not empty, and trivial otherwise. 

Formulas $\phi$ are called {\em TCL} {\em formulas}, {\em sentences}, or 
{\em state formulas}.  
Formulas $\psi$ and $\eta$ are called {\em tree formulas}.  
Note that we strictly require that non-trivial strategy interaction cannot 
cross path modal operators.  
This restriction is important because it offers a sufficient level of locality to efficiently model-check a system against a TCL property.
To illustrate this and to provide a simple extension that offers more expressive power to the cost of a much higher complexity, we informally discuss a small extension,
\emph{extended TCL} (ETCL), where the production rule of $\psi$ also contains $\neg \psi$ and show that it can be used to encode ATL$^*$, and the realisability problem of prenex QPTL can be reduced to ETCL model-checking.

%  and allows us to analyse the interaction 
% of strategies locally in a state and 
% then enforce the interaction along all paths from the state.  


For convenience, we also have the following shorthand notations.
\begin{center}
$\begin{array}{rclcrcl}
\true
& \equiv & p\vee(\neg p) 
& \hspace*{10mm} & % \\
\false
& \equiv & \neg\true \\
\phi_1\wedge \phi_2
& \equiv & \neg((\neg\phi_1)\vee(\neg\phi_2)) 
& \hspace*{10mm} & % \\
\phi_1\Rightarrow \phi_2
& \equiv & (\neg \phi_1)\vee \phi_2 \\
\pevt \phi_1
& \equiv & \true\,\until \phi_1 
& \hspace*{10mm} & % \\
\pfrr \phi_1
& \equiv & \false\,\rmrel \phi_1 \\
% \neg\psi_1\until\psi_2
% & \equiv & \psi_1\rmrel \psi_2 
% & \hspace*{10mm} & % \\
% \neg\psi_1\rmrel\psi_2
% & \equiv & \psi_1\until \psi_2 \\
\neg\nxt \phi_1
& \equiv & \nxt\neg\phi_1 
& \hspace*{10mm} & % \\
\; \langle A\rangle \nxt\psi_1 
& \equiv & \langle A\rangle \langle+\rangle\nxt\psi_1 \\
\; \langle A\rangle \psi_1\until\psi_2
& \equiv & \langle A\rangle\langle+\rangle \psi_1\until\psi_2 
& \hspace*{10mm} & % \\
\; \langle A\rangle \psi_1\rmrel\psi_2  
& \equiv & \langle A\rangle \langle+\rangle\psi_1 \rmrel\psi_2 \\
% & \hspace*{10mm} & % \\
% \; \langle A\rangle \psi_1\until\psi_2
% & \equiv & \langle A\rangle\langle+\rangle \psi_1\until\psi_2 \\
\end{array}
$
\end{center} 
In general, it would also be nice to have the universal SQs and 
SIQs as duals of existential SQs and SIQs, respectively. 
Couldn't we add, or encode by pushing negations to state formulas, 
a property of the form $[+A]\psi_1$, meaning that, for all strategies of agency $A$,  $\psi_1$ will be fulfilled? 
In principle, this is indeed no problem, and extending the semantics would be simple.
This logic would be equivalent to allowing for negations in the production rule of $\psi$.
The problem with this logic is that it is too succinct.
We will briefly discuss in the following section that model-checking becomes non-elementary if we allow for such negations.

% Note, however, that an introduction of negations in the production rule of $\eta$ is only notational sugar.
% As the SIQs in the $\eta$ production rule do not introduce strategies, they are self-dual, and negations can be pushed inside.
% (Cf.\ example (E) from the introduction, where the left hand side of implication can be replaced by
% $( \langle+\emptyset\rangle\pevt \neg \mbox{\tt jail}_a)
% \vee \langle-a\rangle \pfrr \mbox{\tt jail}_a$.)

% All TCL formulas are {\em well-formed} 
% since all their SIQs occur inside
% an SQ beginning with $\langle  A\rangle$
% for some $A$.

From now on, we assume that we are always in the context of a
given TCL sentence. % $\chi$.  
\subsection{semantics}
In order to prepare the definition of a semantics for TCL formulas, 
we start with the definition of a semantics for sentences 
of the form $\langle A \rangle \psi$, where $\psi$ does not contain any SQs.
We call these formulas \emph{primitive} TCL formulas.

Due to the design of TCL, strategy bindings can only effectively happen at non-trivial SQs $\langle A \rangle$ and 
when a non-trivial SIQ $\langle +B\rangle$ is interpreted.
% Moreover, for a subformula $\psi_1\until\psi_2$ 
% (or $\psi_1\rmrel\psi_2$), there is no new strategy binding in 
% $\psi$ to interact with those bound to $\psi_1\until\psi_2$ 
% (or $\psi_1\rmrel\psi_2$).  
%
To ease referring to these strategies, we first define the \emph{bound agency} of a subformulas $\phi$ of a TCL sentence $\chi$, denoted $\embnd(\phi)$, as follows.
\begin{list1} 
\item For state formulas $\phi$, $\embnd(\phi)= \emptyset$.  
\item For state formulas $\langle A \rangle \psi$,  $\embnd(\psi)= A$ (unless $\psi$ is a state formula).
\item For tree formulas $\psi_1=\langle +A \rangle \psi_2$,  $\embnd(\psi_2) = \embnd(\psi_1) \cup A$.
\item For tree formulas $\psi_1=\langle -A \rangle \psi_2$,  $\embnd(\psi_2) = \embnd(\psi_1) \smallsetminus A$.
\item For all other tree formulas $\psi_1$ or $\psi_2$ with $\psi = \psi_1 \mathsf{OP} \psi_2$, with $\mathsf{OP} \in \{\wedge,\vee,\mathcal U, \mathcal R\}$, we have $\embnd(\psi_1) = \embnd(\psi)$ or $\embnd(\psi_2) = \embnd(\psi)$, respectively.
\end{list1} 
$\embnd$ shows, which agents have strategies assigned to them 
by an SIQ or SQ.
Note that this leaves the $\embnd$ undefined for all state formulas not in the scope of an SQ formulas.
For completeness, we could define $\embnd$ as empty in these cases, but a definition will not be required in the definition of the semantics.

As the introduction of additional strategies through non-trivial SIQ $\langle +B\rangle$ is governed by a \emph{positive} 
Boolean combination, all strategy selections can be performed concurrently.
Such a design leads us to the concept of strategy schemes.

A {\em strategy scheme} $\sigma$ is the set of strategies introduced by any non-trivial SQ $\langle A \rangle$ or SIQ $\langle + A \rangle$.
By abuse of notation, we use $\sigma[\phi,a]$ to identify such a strategy.
Read in this way, $\sigma$ can be viewed as a partial function from subformulas and their bound agencies to strategies.
Thus, $\sigma[\phi,a]$ is defined if $a\in \embnd(\phi)$ is in the bound agency of $\phi$.

% The SIQ or SQ operator introducing a strategy for $a$ can be found by walking up the formula tree, starting at $\phi$, and stopping once a formula $\psi$ starting with $\langle A \rangle$ or $\langle + A \rangle$ with $a\in A$ is hit.
% $\sigma[\phi,a]$ refers to the strategy for Agent $a$ introduced by this quantifier.

For example, given a strategy scheme $\sigma$ for a 
TCL sentence $\langle 1\rangle \pevt ((\langle+2\rangle \nxt p)\wedge\langle 2\rangle\pfrr q)$, 
the strategy used in $\sigma$ by Agent $1$ to enforce 
the whole formula can be referred to by
\[\sigma[\langle 1\rangle \pevt ((\langle+2\rangle \nxt p)\wedge\langle 2\rangle\pfrr q),1],\]
but also by $\sigma[\langle+2\rangle \nxt p,1]$, while $\sigma[\langle 2\rangle\pfrr q,1]$ is undefined.
% the one to enforce $\langle+2\rangle \nxt p$ by Agent $2$ is 
% named $\sigma[\langle+2\rangle \nxt p,2]$.  


We use a simple tree semantics for TCL formulas. 
A (computation) tree $T_r$ is obtained by unravelling $\calg$ from $r$  
and expand the ownership 
and labelling functions from $\calg$ to $T_r$ in the natural way.
Technically, we have the following definition.  

\paragraph{\bf Definition: Computation Tree.} 
A {\em computation tree} for a turn based game $\calg$ from a state $q$, 
denoted $T_q$, is the smallest set of play prefixes that contains $q$ and, for all $\pi\in T$ and $(\emlast(\pi),q')\in \cale$, $\pi q'\in T$.
\qed 


The {\em strategy-pruned} tree for a tree node $\pi$, 
a strategy scheme $\sigma$,
and a subformula $\psi_1$ of $\chi$ from a state $q$, 
in symbols $T_q\langle\pi,\sigma,\psi_1\rangle$, 
is the smallest subset of $T_q$ such that: 
\begin{list1} 
\item $\pi\in T_q\langle\pi,\sigma,\psi_1\rangle$;  
\item for all $\pi' \in T_q\langle\pi,\sigma,\psi_1\rangle$ 
  with $\omega\big((\emlast(\pi')\big)\notin\embnd(\psi_1)$ 
  and $(\emlast(\pi'),q')\in \cale$, 
  $\pi' q'\in T_q\langle\pi,\sigma,\psi_1\rangle$;  
\item for all $\pi' \in T_q\langle\pi,\sigma,\psi_1\rangle$,  
  $a=\omega\big((\emlast(\pi')\big)$, and 
  $q'=\sigma[\psi_1,a](\pi')$ 
  with $a\in\embnd(\psi_1)$, 
%   $(\emlast(\pi),q') \in \cale$ and 
  $\pi' q' \in T_q\langle\pi,\sigma,\psi_1\rangle$.
\end{list1}
Given a computation tree or a strategy-pruned tree $T$ and a node $\pi\in T$, 
for every $\pi q\in T$, we say that $\pi q$ is a successor of $\pi$ in $T$. 
A play $\rho$ is a \emph{limit of $T$} (or an infinite path in $T$), in symbols $\rho\stackrel{\infty}\in T$, 
if there are infinitely many prefixes of $\rho$ in $T$. 


We now define the semantics of subformulas of primitive TCL formulas 
inductively as follows.  
Given the computation tree $T_q$ of $\calg$, a tree node $\pi\in T_q$, 
and a strategy scheme $\sigma$, 
we write $T_q,\pi,\sigma\models\psi_1$ to denote that 
$T_q$ satisfies $\psi_1$ at node $\pi$ with strategy scheme $\sigma$.

While the notation might seem heavy on first glance, note that the truth for state formulas merely depends on the state $\emlast(\pi)$ in which they are interpreted, and the tree formulas are simply interpreted on a strategy pruned tree rooted in $\pi$ and defined by the strategy scheme.
\begin{list1}
\item For state formulas $\phi$ other than SQ formulas,
  we use the state formula semantics:
$T_q,\pi,\sigma\models \phi$ iff $\calg,\emlast(\pi) \models \phi$, 
with the usual definition.  
\begin{list2}
\item  $\calg,q \models p$ if, and only if, $p\in\lambda(q)$,
\item  $\calg,q \models \neg \phi$ if, and only if, $\calg,q \not\models \phi$,
\item  $\calg,q \models \phi_1 \vee \phi_2$ if, and only if, $\calg,q \models \phi_1$ or $\calg,q \models \phi_2$, and
\item  $\calg,q \models \phi_1 \wedge \phi_2$ if, and only if, $\calg,q \models \phi_1$ and $\calg,q \models \phi_2$.
\end{list2}
(Note that this allows for using negation for state formulas.)

\item $T_q,\pi,\sigma \models \psi_1\vee\psi_2$ iff
    $T_q,\pi,\sigma \models \psi_1$
    or $T_q,\pi,\sigma \models \psi_2$.
     \mbox{(The $\psi_i$ are no state formulas.)}
\item $T_q,\pi,\sigma\models \psi_1\wedge\psi_2$ iff
    $T_q,\pi,\sigma\models \psi_1$
    and $T_q,\pi,\sigma\models \psi_2$ hold.
    
\item $T_q,\pi,\sigma\models \langle \pm A \rangle \nxt \psi$ iff,
     for all successors $\pi q'$ of $\pi$ 
     in $T_q\langle \pi,\sigma,\langle \pm A \rangle\nxt\psi_1\rangle$, 
     $T_q,\pi q',\sigma \models \psi$ holds.  
    
\item $T_q,\pi,\sigma \models \langle \pm A \rangle \psi_1\until\psi_2$ iff, for all limits 
    $\rho\stackrel{\infty}\in T_q\langle \pi,\sigma, \langle \pm A \rangle\psi_1\until\psi_2\rangle$, 
    there is a $k \geq |\pi|-1$ 
    such that $T_q,\rho[0,k],\sigma \models \psi_2$ and, 
    for all $h\in [|\pi|-1,k-1]$, 
    $T_q,\rho[0,h],\sigma\models \psi_1$ hold. 
\item $T_q,\pi,\sigma \models \langle \pm A \rangle \psi_1\rmrel\psi_2$ iff, 
    for all limits 
    $\rho\stackrel{\infty}\in T_q\langle\pi,\sigma,\langle \pm A \rangle\psi_1\rmrel\psi_2\rangle$, one of the following two restrictions 
    are satisfied. 
    \begin{list2} 
    \item For all $k \geq |\pi|-1$, $T_q,\rho[0,k],\sigma \models \psi_2$. 
    \item There is a $k\geq |\pi|-1$ such that 
      $T_q,\rho[0,k],\sigma\models \psi_1\wedge\psi_2$, and, 
      for all $h\in [|\pi|-1,k]$, $T_q,\rho[0,h],\sigma\models \psi_2$.
    \end{list2} 
\item $T_q,\pi,\sigma\models \langle \pm A \rangle \psi_1$ iff
    $T_q,\pi,\sigma \models \psi_1$.
\item $\calg, q \models \langle A \rangle \psi_1$ iff there is a strategy scheme $\sigma$ such that
    $T_q,q,\sigma \models \psi_1$.
\end{list1}

If $\phi_1$ is a TCL sentence %and $T_q,q,\sigma\models\phi_1$ 
% for some strategy scheme $\sigma$,
% then we may simply write $\calg,q\models\phi_1$.  
% If
then we write $\calg\models\phi_1$ for $\calg,r\models\phi_1$.


Note that, while asking for the existence of a strategy scheme refers to all strategies introduced by some SQ or SIQ in the TCL sentence, only the strategies introduced by the respective SQ and the SIQs in its scope are relevant.

The simplicity of the semantics is owed to the fact that 
it suffices to introduce new strategies at the points 
where eventualities become true for the first time.
Thus, they do not really depend on the position 
in which they are invoked and we can guess them up-front. 
(Or, similarly, together with the points on the unravelling 
where they are invoked.)
%
This is possible, simply because the validity of state formulas (and hence of TCL sentences) cannot depend on the validity of the left hand side of an until (or the right hand side of a release) \emph{after} the first time it has been satisfied. 


\section{Expressive Power of TCL}
Note that TCL is not a superclass of BSIL since 
BSIL allows for negation in front of SIQs while TCL does not. 
However, by examining the proofs in \cite{WHY11} 
for the inexpressibility of BSIL properties by ATL$^*$, GL, and AMC, 
we find that the BSIL sentence used in the proofs is also a TCL sentence. 
This leads to the conclusion that 
there are properties expressible in TCL but cannot 
be expressed in ATL$^*$, GL, and AMC.  

{\lemma \label{lemma.express.incomp1} 
There are TCL sentences that cannot be expressed 
in any of ATL$^*$, GL, or AMC.  
} 
\qed 


TCL is, in fact, not only a powerful logic, 
but also contains important logics either as syntactical fragments or can embed them in a straight forward way.
ATL and CTL can be viewed as syntactic fragments of TCL.

But it is also simple to embed LTL and even CTL$^*$.
We start with $\exists$LTL, the less used variant 
where one is content if one path satisfies the formula.
% Let $A$ denote the set of all agents.
We then translate an LTL formula, which we assume w.l.o.g.\ 
to be in negative normal form 
(negations only in front of atomic propositions).
Then ``there is a path that satisfies $\phi$'' is equivalent to
$\langle 1,\ldots,m \rangle \widehat{\phi}$, where $\widehat{\phi}$ is derived from $\phi$ by replacing every occurrence of $\nxt$, $\until$, and $\rmrel$ by
$\langle + \rangle \nxt$, $\langle + \rangle \until$, and $\langle + \rangle \rmrel$, respectively.
%
The simple translation is possible because the formula $\widehat{\psi}$ 
is de-facto interpreted over a path, 
the path formed by the joint strategy of the agency $[1,m]$.
The $\langle + \rangle$ operators we have added have no effect on the semantics in such a case, just as a CTL formula can be interpreted as the LTL formula obtained by deleting all path quantifiers when interpreted over a word.

Consequently, we have the expected semantics for $\forall LTL$:  ``all paths satisfy $\phi$'' is equivalent to
$\neg \langle A \rangle \widehat{\neg\phi}$, where $\neg \phi$ is assumed to be re-written in negative normal form.
% This immediately implies that TCL model-checking is PSPACE hard.
% 
The encoding of $\exists$LTL and $\forall$LTL can easily be extended to the encoding of CTL$^*$.  
%
{\lemma \label{lemma.express.ctl.star}
TCL is more expressive than CTL$^*$ and LTL.  
}
\qed 

This encoding does not extend to ATL$^*$.
$\langle 1\rangle ((\pfrr p)\vee \pfrr q)$  is an ATL$^*$ property that cannot be 
expressed with TCL.

This is different from the 
% \begin{center} 
ATL property $(\langle 1\rangle \pfrr p)\vee \langle 1\rangle\pfrr q$ 
% \end{center} 
or the
% \begin{center} 
TCL property
\mbox{$\langle 1\rangle ((\langle+\rangle\pfrr p)\vee \langle +\rangle\pfrr q)$}. 
% \end{center} 
In fact, the proofs and examples in \cite{WHY11} 
can also be applied in this work to show that 
there are properties of ATL$^*$ (or GL, or AMC) 
that cannot be expressed with TCL.  
This leads to the following lemma. 

{\lemma \label{lemma.express.incomp2} 
TCL is incomparable in expressiveness with ATL$^*$, GL, and AMC.  
} 
\qed 

Note, however, that allowing for a negation in the definition of $\psi$ would change the situation.
Then an ATL$^*$ formula $\langle A \rangle \psi$ 
(assuming for the sake of simplicity that $\psi$ is an LTL formula), 
would become
$\langle A \rangle \neg \langle + [1,m] \smallsetminus A \rangle \widehat{\neg \psi}$ in the extended version of TCL.
The translation extends to full ATL$^*$, but this example also demonstrates why negation is banned: even without nesting, we can, by encoding ATL$^*$, encode a 2EXPTIME complete model-checking problem, losing the appealing tractability of our logic.

In fact, it is easy to reduce the realisability problem of prenex QPTL, and hence a non-elementary problem, to the model-checking problem of extended TCL.
Using the game structure from Figure \ref{fig.nonelementary}, we can encode the realisability of a prenex QPTL formula with $n-1$ variables, for simplicity of the form
$\forall p_2 \exists p_3 \forall p_4 \ldots \exists p_n \phi$, where $p_2,\ldots,p_n$ are all propositions occurring in $\phi$.
We reduce this to model-checking the formula 
\[\phi'=\langle 1 \rangle \neg \langle +2 \rangle \neg \langle +3 \rangle \neg \langle +4 \rangle \neg \ldots \neg \langle +n \rangle (\psi_\phi \wedge \langle + \rangle \pfrr p_1),\]
where $\psi_\phi$ can be obtained from $\widehat{\phi}$ by replacing
\begin{list1}
\item every literal $p_i$ by $\langle - 1 \rangle \langle + 1 \rangle \nxt (p_i \wedge \langle + \rangle \nxt p_i)$, and
\item every literal $\neg p_i$ by $\langle - 1 \rangle \langle + 1 \rangle \nxt (p_i \wedge \langle + \rangle \nxt \neg p_i)$.
\end{list1}

These formulas are technically not extended TCL formulas as $\langle +i \rangle \psi_1$ is not part of the production rule of $\psi$,  but $\langle +i \rangle \psi_1$ can be used as an abbreviation for $\langle +i \rangle \false \until \psi_1$.

Checking satisfiability of $\phi$ is is equivalent to model-checking $\phi'$ on the game shown in Figure \ref{fig.nonelementary}. 
The game has $n+1$ nodes, agents, and atomic propositions.
The nodes in Figure \ref{fig.nonelementary} are labeled with the agent that owned the nodes, and the atomic proposition $p_i$ is true exactly in node $i$.
From his state, Agent $1$ can move to any other state, 
while all other agents can either stay in their state or return 
to the state owned by Agent $1$.

The game starts in the node owned by Agent $1$, and in order to comply with the specification, the outermost strategy profile chosen by Agent $1$ must be to stay in the initial state for ever.
$\psi_\phi$ is chosen to align the truth of $p_i$ at position $j \in \mathbb N$ with the decision that Agent $i$ makes on the history $1^j i$: $\true$ corresponds to staying in $i$ and $\false$ with returning to $1$.

\begin{figure}[t]
{
\begin{center}
\psset{xunit=2cm,yunit=-1cm}
\begin{pspicture}(0,0)(5,1.2)
\tiny
\pnode(2.25,0){1}
\rput(2.5,0){\circlenode{0}{$1$}}

\ncline{->}{1}{0}

\rput(0,1){\circlenode{1}{$2$}}
\rput(1,1){\circlenode{2}{$3$}}
\rput(2,1){\circlenode{3}{$4$}}
\rput(3,1){\circlenode{4}{$5$}}
\rput(4,1){$\cdots$}
\rput(5,1){\circlenode{5}{$n$}}

\ncline{<->}{0}{1}
\ncline{<->}{0}{2}
\ncline{<->}{0}{3}
\ncline{<->}{0}{4}
\ncline{<->}{0}{5}

\nccircle[angleA=0]{->}{0}{.18}
\nccircle[angleA=180]{->}{1}{.18}
\nccircle[angleA=180]{->}{2}{.18}
\nccircle[angleA=180]{->}{3}{.18}
\nccircle[angleA=180]{->}{4}{.18}
\nccircle[angleA=180]{->}{5}{.18}

\end{pspicture}
\end{center}
}
\caption{The turn-based game graph from the non-elementary hardness proof of extended TCL.}
\label{fig.nonelementary}
\end{figure} 

It is not hard to establish a matching upper bound for model-checking extended~TCL.

\section{Complexity of TCL\label{sec.gen.proc}} 

In this section, we show that model-checking TCL formulas is EXPTIME-complete in the formula and P-complete in the model (and for fixed formulas), while the satisfiability problem is 2EXPTIME-complete.
As the proof of inclusion of the satisfiability problem in 2EXPTIME builds on the proof of the inclusion of model-checking in EXPTIME, we start with an outline of the EXPTIME hardness argument for the TCL model-checking problem and then continue with describing EXPTIME and 2EXPTIME decision procedures for the TCL model and satisfiability checking problem, respectively.
2EXPTIME hardness for TCL satisfiability is implied by the inclusion of CTL* as a de-facto sub-language \cite{Vardi+Stockmeyer/85/CTL}.



We show EXPTIME hardness by a reduction from 
the {\sf PEEK-$G_6$}~\cite{SC79} game.
An instance of {\sf PEEK-$G_6$} consists of 
two disjoint sets of boolean variables, 
$P_1=\{p_1,\ldots,p_h\}$ (owned by a safety agent) and 
$P_2=\{p_{h+1},\ldots,p_{h+k}\}$ (owned by a reachability agent), 
a subset $I\subseteq P_1\cup P_2$ of them that are initially {\em true}, 
and a boolean formula $\gamma$ in CNF over 
$P_1\cup P_2$ that the reachability agent wants to become 
{\em true} eventually.
The game is played in turns between the safety and the reachability agent 
(say, with the safety agent moving first), and 
each player can change the truth value of one of his or her variables 
in his/her turn.

\begin{lemma}
\label{lemma.exptime.hard}
TCL model-checking is EXPTIME hard for primitive TCL formulas.
\end{lemma}

\begin{proof}
To reduce determining the winner of an instance of a {\sf PEEK-$G_6$} game to TCL model-checking, 
we introduce a 2-agent game $\calg=\langle 2,\calq,r,\omega,\calp,\lambda,\cale\rangle$ 
as shown in Figure \ref{fig.exptime}, where Agent 1 (he, for convenience) represents the safety agent 
while Agent 2 (she, for convenience) represents the reachability agent.
$t_{h+k}$ and $f_{h+k}$ are the only states owned by Agent 2.

\begin{figure}[t]
{
\begin{center}
\psset{xunit=2cm,yunit=1cm}
\begin{pspicture}(0,-.7)(5,1.7)
\tiny
\pnode(-.25,0.5){0}
\rput(0,0.5){\circlenode{r}{$r$}}

\rput(1,0){\circlenode{t1}{$f_1$}}
\rput(1,1){\circlenode{f1}{$t_1$}}

\rput(2,0){\circlenode{t2}{$f_2$}}
\rput(2,1){\circlenode{f2}{$t_2$}}

\rput(3,0){\circlenode{t3}{$f_3$}}
\rput(3,1){\circlenode{f3}{$t_3$}}

\ncline{->}{0}{r}
\ncline{->}{r}{t1}
\ncline{->}{r}{f1}

\ncline{->}{t1}{t2}
\ncline{->}{t1}{f2}
\ncline{->}{f1}{t2}
\ncline{->}{f1}{f2}

\ncline{->}{t2}{t3}
\ncline{->}{t2}{f3}
\ncline{->}{f2}{t3}
\ncline{->}{f2}{f3}

\rput(3.5,0.5){$\cdots$}

\rput(4,0){\rnode{t1}{\psframebox[framesep=4pt]{$f_{h+k}$}}}
\rput(4,1){\rnode{f1}{\psframebox[framesep=4pt]{$t_{h+k}$}}}

\rput(5,-.5){\circlenode{t2}{$1$}}
\rput(5,0.5){$\vdots$}
\rput(5,1.5){\circlenode{f2}{$k$}}

\ncline{->}{t1}{t2}
\ncline{->}{t1}{f2}
\ncline{->}{f1}{t2}
\ncline{->}{f1}{f2}

\ncarc[arcangle=-20.5]{->}{f2}{r}
\ncarc[arcangle=20.5]{->}{t2}{r}

\end{pspicture}
\end{center}
}
\caption{The turn-based game graph from the EXPTIME hardness proof.}
\label{fig.exptime}
\end{figure} 

The game is played in rounds, and a round starts each time the game is at state $r$.
If the game goes through $t_i$ this is identified with the variable $p_i$ to be true.
Likewise, going through $f_i$ is identified with the variable being false.

It is simple to write a TCL specification that forces the safety player to toggle the value of exactly one of his variables in each round, and to toggle the value of the variable $p_{h+i}$ of the reachability player defined by the state $i$ she has previously moved to, while maintaining all other variable values.
Requiring additionally that the safety agent can guarantee that the boolean formula is never satisfied provides the reduction.
\qed
\end{proof}

The details of the construction are available in the full version. 
It is interesting that a game with only two agents suffices for the proof.  
Two agents are also sufficient to show P hardness for fixed formulas, as solving a reachability problem for AND-OR graphs \cite{Immerman81} naturally reduces to showing $\langle 1 \rangle \pevt p$.

\begin{lemma}
\label{lemma.nl.hard}
TCL model-checking for fixed formulas is P hard for primitive TCL formulas.
\qed 
\end{lemma}

In order to establish inclusion in EXPTIME and P, respectively, we use an automata based argument.

\begin{theorem}
\label{theo.ms.complete}
The model-checking problem of TCL formulas against turn-based game graphs
is EXPTIME-complete, and P-complete for fixed formulas.
\end{theorem}

\begin{proof}
We first show the claim for primitive TCL formulas $\phi=\langle A \rangle \psi$.
 
To keep the proof simple, we first consider a tree automaton $\mathcal U$ that checks the acceptance of $\psi$ for a given strategy scheme $\sigma$.
That is, $\mathcal U$ checks if ${T_q}^+,q,\sigma \models \psi$ under the assumption that both $\sigma$ and the truth values for the subformulas starting with a $\langle \pm B \rangle$ are encoded in the nodes of ${T_q}^+$.

Such an automaton would merely have to run simple consistency checks, and it is simple to construct a suitable universal weak tree automaton $\mathcal U$, which is polynomial in the size of $\phi$.
From there it is simple to infer a deterministic B\"uchi tree automaton 
$\mathcal D$, which is exponential in the weak universal tree automaton 
\cite{Muller+Schupp/95/Alternating}.

It is then a trivial step (projection) to \emph{guess} $\sigma$ and the truth annotation of the subformulas on the fly, turning the deterministic B\"uchi tree automaton $\mathcal D$ that requires a correct annotation into a nondeterministic B\"uchi automaton $\mathcal N$ of the same size that checks $\calg,q \models \phi$.
%
Acceptance can be checked in time quadratic in the size of the product of $\mathcal N$ and $\calg$ \cite{Chatterjee+Henzinger/12/Buchi}.

To take the step to full TCL, we can model-check the truth of primitive TCL formulas and then use the result of this model-checking instead of the respective subformula.

Hardness is inherited from Lemmata \ref{lemma.exptime.hard} and \ref{lemma.nl.hard}.
\qed
\end{proof}

This argument shows more: the complexity of TCL model-checking for fixed formulas does not depend on the formula.
It suffices to solve a number of B\"uchi games, where both the size of the game and the number of games to be played is linear in $\mathcal G$.

\begin{corollary}
Viewing the size of a TCL sentence as a parameter, TCL model-checking is fixed parameter tractable.
\end{corollary}

The automata construction from the proof of Theorem \ref{theo.ms.complete} extends to a construction for satisfiability checking.

