\begin{abstractEN}

This thesis is constructed by 3 parts. 
In the first part, I introduce an algorithm to calculate the highest degree of fault tolerance a system can achieve with the control of a safety critical systems. Which can be reduced to solving a game between a malicious environment and a controller. During the game play, the environment tries to break the system through injecting failures while the controller tries to keep the system safe by making correct decisions. I found a new control objective which offers a better balance between complexity and precision for such systems: we seek systems that are k-resilient. A system is k-resilient means it is able to rapidly recover from a sequence of small number, up to k, of local faults infinitely many times if the blocks of up to k faults are separated by short recovery periods in which no fault occurs. k-resilience is a simple abstraction from the precise distribution of local faults, but I believe it is much more refined than the traditional objective to maximize the number of local faults. I will provide detail argument of why this is the right level of abstraction for safety critical systems when local faults are few and far between. I have proved, with respect to resilience, the computational complexity of constructing optimal control is low. And a demonstration of the feasibility through an implementation and experimental results will be in following chapters.
The second part is to create an logic which can describe the different purposes of each player such as environment, controller, user, and etc in a system. I propose an extension to ATL (alternating-time logic), called BSIL(basic strategy-interaction logic), for the specification of strategies interaction of players in a system. BSIL is able to describe one system strategy that can cooperate with several strategies of the environment for different requirements. Such properties are important in practice and I show that such properties are not expressible in ATL*, GL (game logic), and AMC (alternating μ-calculus). Specifically, BSIL is more expressive than ATL but incomparable with ATL*, GL, and AMC in expressiveness. I show that, for fulfilling a specification in BSIL, a memoryful strategy is necessary. I also show that the model-checking complexity of BSIL is PSPACE-complete and is of lower complexity than those of ATL*, GL, AMC, and the general strategy logics. Which may imply that BSIL can be useful in closing the gap between large scale real-world projects and the time consuming game-theoretical results. I then show the feasibility of our techniques by implementation and experiment with our PSPACE model-checking algorithm for BSIL.
The final part is an extension to BSIL called temporal cooperation logic(TCL). TCL allows successive definition of strategies for agents and agencies. Like BSIL the expressiveness of TCL is still incomparable with ATL*, GL and AMC. However, it can describe deterministic Nash equilibria while BSIL cannot. I prove that the model checking complexity of TCL is EXPTIME-complete. TCL enjoys this relatively cheap complexity by disallowing a too close entanglement between cooperation and competition while allowing such entanglement leads to a non-elementary complexity. I have implemented a model-checker for TCL and shown the feasibility of model checking in the experiment on some benchmarks.  


Key words:

\end{abstractEN}
